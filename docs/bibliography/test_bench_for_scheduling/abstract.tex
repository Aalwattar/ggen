Générateur de graphes aléatoires: Standard-Task-Graph (STG);
Udo Hönig, Wolfram Schiffmann;


Le générateur de graphes prendre comme paramètres d'entrée:

1. La taille du graphe: n
2. Le degré 
3. La distance entre nodes
4. Le temps de calculs des nodes
5. Le temps de communication

Il utilise une loi de distribution normal.

En considérant le degré, le générateur génère 4 différents ensembles des graphes selon son degré.
Pour le premier ensemble (faible degré) chaque node est connecté à un 25\% (d'espérance) de toutes les nodes avec indice supérieur.
Pour l'ensemble a degré moyen, l'espérance mathématique est de 50\%.
Pour l'ensemble forte degré,  l'espérance mathématique est de 75\%
Dans toutes les cas l'écart type est à 25\%.
Pour éviter les biais, le quatrième ensemble contient des graphes générés avec un degré aléatoire entre 1\% et 100\% . Pour cela le degré es calculé en utilisant une distribution uniforme.

La distance est calculé de la même façon que pour le degré avec les mêmes probabilités.

Pour le temps de calcul des nodes et le temps de communication, il y a trois valeurs différentes: pour la première valeur  l'espérance est égale à 15 unités de temps dont l'écart type est de 3 unités de temps, pour la deuxième valeur  l'espérance est de 5 avec un écart type de 3. Il utilise une loi de distribution normal pour générer les valeurs. La troisième valeur  étant tiré uniformément distribué entre 1 et 20.
